\documentclass[]{article}

\usepackage{graphicx}
\usepackage[sort,round]{natbib}
\usepackage[dvipsnames,usenames]{color}
\usepackage{amsmath}
\usepackage{amssymb}
\usepackage{wasysym}
\usepackage[linkcolor=Sepia,citecolor=Sepia,colorlinks=true]{hyperref}
\usepackage[all]{hypcap}

% Some useful commands and shorthand macros
\newcommand{\unit}[1]{\;\mathrm{#1}}
\newcommand{\sub}[1]{_{\text{#1}}}
\newcommand{\about}{\sim\!}
\newcommand{\V}[1]{\mathbf{#1}}
\newcommand{\code}[1]{\textsc{#1}}
\newcommand{\eps}{\varepsilon}
\newcommand{\Sun}{\astrosun}
\newcommand{\di}{\partial} % partial derivative
\newcommand{\kros}{\times} % vector product
\newcommand{\grad}{\boldsymbol{\nabla}} % gradient operator
\newcommand{\divergenz}[1]{\boldsymbol{\nabla}\cdot\V{#1}} % divergence operator
\newcommand{\curl}[1]{\boldsymbol{\nabla}\kros\V{#1}} % curl (rot) operator

\begin{document}
\title{Stellar Physics}
\author{Naor Movshovitz}
\maketitle

\begin{abstract}
Notes on stellar physics, summarized from Dina's book.
\end{abstract}

\tableofcontents

\section{Observational background and basic assumptions}
\subsection{What is a star?}
\begin{itemize}
\item A \emph{star} is an object that satisfies two conditions: (a) it is
bound by self-gravity; (b) it radiates energy supplied by an internal source.

\item From the first condition it follows that the shape of a star should be
spherical, since gravity is a spherically symetric force field.

\item The source of radiation is usualy nuclear energy released by fusion
reactions that take place in the interior of the star, and sometimes
gravitational potential energy released in contraction or collapse.
\end{itemize}

\subsection{What can we learn from observations?}
\begin{itemize}
\item The primary measurable quantity of a star is its \emph{apparent
brightness}, which is the amount of radiation, per unit time, falling on a unit
area of a collector (a telescope). This radiation flux, denoted by
$I_{\text{obs}}$, is not an intrinsic property of the star, because it depends
on the distance of the star from the observer. The intrinsic property is the
\emph{luminosity} $L$, defined as the amount of energy radiated by the star per
unit time. If $d$ is the distance of the observer then the relationship between
the luminosity and the apparent brightness is

\begin{equation}
I_{\text{obs}}=\frac{L}{4\pi{d}^2}.
\end{equation}

The luminosity of a star is usually expressed relative to that of the Sun, in
units of \emph{solar luminosity}, $L_{\Sun}=3.85\times10^{26}\unit{J\,s^{-1}}$.
Stellar luminosities range between $10^{-5}L_{\Sun}$ and $10^5L_{\Sun}$.

\item The only \emph{direct} method of determining distance to stars is based on
the old concept \emph{parallax} -- the angle between the lines of sight of the
star from two different positions of the observer. The further the observed
object is, the wider the baseline needs to be, to obtain a discernible parallax.
For objects in the solar system, distance points on the Earth can be used. For
stars, the baseline is supplied by the Earth's orbit around the Sun, giving a
distance of $2\unit{au}\approx3\times10^{11}\unit{m}$. The star's position is
determined relative to very distant, fixed stars, at an interval of half a year.
The triangle obtained is very nearly isosceles; the parallax, defined as half
the apex angle, is less than $1''$. The distance to the star is then $d=1/p$, in
astronomical units. In this method, distances of up to 500 light-years can be
measured. The \emph{parsec} is the distance corresponding to a parallax of one
arc second, amounting to about 3 light-years.

\item The surface temperature of a star can be obtained from the shape of its
spectrum -- the \emph{continuum}. The \emph{effective temperature} of a star,
$T_\text{eff}$, is defined as the temperature of a blackbody that would emit the
same radiation flux. This is a good approximation to the temperature of the
star's outermost layer, called the \emph{photosphere}. If the stellar radius is
$R$ then

\begin{equation}\label{eq:LTeff}
L=4\pi{R}^2\sigma{T}_\text{eff}^4.
\end{equation}

The surface temperatures of stars range between a few thousand to a few hundred
thousands degrees Kelvin, the wavelength of maximum radiation shifting from
infrared to X-ray. The effective temperature of the Sun is $5780\unit{K}$.

\item The chemical composition can be inferred from spectral lines, but these
will only reveal elements present in the photosphere, and the deduced
composition is not representative of the bulk. Most of chemical elements were
found present in the spectrum of stars.

\item Under certain conditions, the mass of a star that is a member of a binary
system can be calculated, based on periodic shifts in the spectral lines, that
can be translated into an orbital period. The radius of a star can very rarely
be measured in eclipsing binary systems. More often, the star's radius is
determined from the independently measured luminosity and effective temperature,
using Eq.~\eqref{eq:LTeff}. Stellar masses and radii are measured in units of
\emph{solar mass}, $M_{\Sun}=1.99\times10^{30}\unit{kg}$, and \emph{solar
radius}, $R_{\Sun}=6.96\times10^8\unit{m}$. The mass range is quite narrow --
between $\sim{0.1}$ and a few tens of $M_{\Sun}$. Stellar radii vary typically
between less than $0.01$ to more than $1000 R_{\Sun}$, but more compact stars
exist, with radii of a few tens of kilometers.

\end{itemize}

\subsection{Basic assumptions}
\begin{description}

\item[Isolation] We regard a single star as an isolated body in empty space.
Although a star is always a member of a larger group (a galaxy, a stellar
cluster) this assumption is justified because the typical distance between stars
is on the order of a few light-years. The Sun's closest neighbor (\emph{Proxima
Centauri}) is at a distance of 4.22 light-years. this distance is larger than
the solar diameter by a factor of $\sim{3}\times{10^7}$. The gravitational field
and radiation flux would be diminished by a factor of almost $10^{15}$.

The assumption of isolation means that the course of a star's evolution is
determined only by initial conditions.

\item[Uniform initial composition] A star is born with a given mass and a given
composition. The compositions of stars, as derived from spectral measurements,
are remarkably similar, and are also similar to the composition of the
interstellar medium. As stars are born in the interstellar medium, we may assume
that there is little difference in the \emph{initial} composition of stars.

A newly formed star is composed out of hydrogen (about 70\% by mass), helium
(25--30\%), and traces of other materials, most abundant of which are oxygen,
carbon, and nitrogen. The composition of stellar material is usually described
in \emph{mass fractions} of elements: $X$ denotes the mass fraction of hydrogen,
$Y$ denotes the mass fraction of helium, and $Z$ denotes the mass fraction of
the other elements, so $X+Y+Z=1$.

The assumption of uniform initial composition means that the evolution of a star
is solely dependent on its initial mass -- stars are a one-parameter family.

\item[Spherical symmetry] Departure from spherical symmetry can be caused by
rotation or by the star's own magnetic field. In the majority of cases, the
energy associated with these factors is much smaller than the gravitational
binding energy. The period of the Sun's revolution is about 27 days. This
corresponds to an angular velocity of $\sim{2.7}\times10^{-6}\unit{rad/s}$. The
kinetic energy of rotation relative to the gravitational binding energy is
\begin{equation*}
\frac{M\omega^2R^2}{GM^2/R}=\frac{\omega^2R^3}{GM}\approx2\times10^{-5}.
\end{equation*}
(The spin velocity of other stars can be deduced by the broadening of spectral
lines caused by the Doppler effect.)

The magnetic fields of stars similar to the Sun range from a few thousandths to
a few tenths of a tesla. Magnetic fields can be deduced from splitting of
spectral lines due to the Zeeman effect. The energy density associated with a
magnetic field $B$ is $B^2/2\mu_0$, while the gravitational energy density is
$GM^2/R^2$. For the Sun, taking $B=0.1\unit{T}$ (typical of sunspots) we get
\begin{equation*}
\frac{B^2/\mu_0}{GM^2/R^4}\approx10^{-11}.
\end{equation*}
Compact stars have larger magnetic fields, but their small radii compensate for
this.

Neglecting Magnetic or rotational effects means neglecting also deviations from
spherical symmetry. The physical properties within a star will change as a
function of radial distance only. The spatial variable $r$ can be replaced by
the mass $m$ enclosed within a volume of radius $r$. The transformation is given
in terms of the density $\rho$:
\begin{equation}
dm=\rho\,4\pi{r}^2dr.
\end{equation}
The advantage of using $m$ as the independent variable is that its range is
bounded, ($0\leq{m}\leq{M}$), whereas the radius can change by several orders of
magnitude in the lifetime of a star.

\end{description}

\subsection{The H--R diagram}
The fundamental properties of a star that can be inferred from observation are
its luminosity and effective temperature. Plotting the logarithm of the
luminosity, in solar luminosity units, against the decreasing logarithm of the
effective temperature in degrees Kelvin is known as the H-R diagram (Ejnar
Hertzsprung 1911 and Henry Russell 1913). After populating the H--R diagram with
many observable stars it becomes obvious that stars occupy only certain areas in
the $L$--$T_\text{eff}$ plane. Most stars lie in the diagonal strip known as the
\emph{main sequence}. Another populated area is to the right and above of the
main sequence. Stars in this region are brighter than main sequence stars of the
same temperature, or put another way, they are cooler than main sequence stars
of the same brightness. For this reason they are called \emph{red giants}. A
third populated region lies to the left and below the main sequence. Stars in
this region are then hotter than main sequence stars of the same brightness, and
are therefore called \emph{white dwarfs} (not dwarves).

Looking at H--R diagrams of different star clusters it is possible to learn
something about the evolution of stars. Stars within a cluster are assumed to
have formed at roughly the same time, by fragmentation of a gas cloud, and their
diagram is therefore a kind of snapshot picture of a moment in the life of the
group. From several of these snapshots we can deduce that being on or outside
of the main sequence is a determined by age, whereas the location along the main
sequence is determined `at birth', i.e., determined by initial mass only.

Choosing main sequence stars with known masses and looking for a correlation
between their masses and luminosities, an empirical relation,
\begin{equation}
\frac{L}{L_{\Sun}}=\left(\frac{M}{M_{\Sun}}\right)^\nu,
\end{equation}
is found, with $\nu$ between 3 and 5. (The Sun is a main-sequence star.)

\section{The equations of stellar evolution}

\end{document}