%\documentclass[]{article}
\documentclass[10pt,amsmath,amssymb,aps,pra]{revtex4-2}

\usepackage{graphicx}
\usepackage[dvipsnames,usenames]{color}
%\usepackage{amsmath}
%\usepackage{amssymb}
\usepackage{wasysym}
\usepackage[linkcolor=Sepia,citecolor=Sepia,colorlinks=true]{hyperref}
\usepackage[all]{hypcap}

% Some useful commands and shorthand macros
\newcommand{\unit}[1]{\;\mathrm{#1}}
\newcommand{\sub}[1]{_{\text{#1}}}
\newcommand{\about}{\sim\!}
\newcommand{\V}[1]{\mathbf{#1}}
\newcommand{\code}[1]{\textsc{#1}}
\newcommand{\eps}{\varepsilon}
\newcommand{\Sun}{\astrosun}
\newcommand{\rsun}{R_{\astrosun}}
\newcommand{\msun}{M_{\astrosun}}
\newcommand{\lsun}{L_{\astrosun}}
\newcommand{\di}{\partial} % partial derivative
\newcommand{\kros}{\times} % vector product
\newcommand{\grad}{\boldsymbol{\nabla}} % gradient operator
\newcommand{\divergenz}[1]{\boldsymbol{\nabla}\cdot\V{#1}} % divergence operator
\newcommand{\curl}[1]{\boldsymbol{\nabla}\kros\V{#1}} % curl (rot) operator

\begin{document}
\title{Stellar Physics}
\author{Naor Movshovitz}
\affiliation{UC Santa Cruz}

\begin{abstract}
Notes on stellar physics, summarized from Dina's book.
\end{abstract}

\maketitle

\tableofcontents

\section{Observational background and basic assumptions}
\subsection{What is a star?}
\begin{itemize}
\item A \emph{star} is an object that satisfies two conditions: (a) it is
bound by self-gravity; (b) it radiates energy supplied by an internal source.

\item From the first condition it follows that the shape of a star should be
spherical, since gravity is a spherically symetric force field.

\item The source of radiation is usualy nuclear energy released by fusion
reactions that take place in the interior of the star, and sometimes
gravitational potential energy released in contraction or collapse.
\end{itemize}

\subsection{What can we learn from observations?}
\begin{itemize}
\item The primary measurable quantity of a star is its \emph{apparent
brightness}, which is the amount of radiation, per unit time, falling on a unit
area of a collector (a telescope). This radiation flux, denoted by
$I_{\text{obs}}$, is not an intrinsic property of the star, because it depends
on the distance of the star from the observer. The intrinsic property is the
\emph{luminosity} $L$, defined as the amount of energy radiated by the star per
unit time. If $d$ is the distance of the observer then the relationship between
the luminosity and the apparent brightness is

\begin{equation}
I_{\text{obs}}=\frac{L}{4\pi{d}^2}.
\end{equation}

The luminosity of a star is usually expressed relative to that of the Sun, in
units of \emph{solar luminosity}, $L_{\Sun}=3.85\times10^{26}\unit{J\,s^{-1}}$.
Stellar luminosities range between $10^{-5}L_{\Sun}$ and $10^5L_{\Sun}$.

\item The only \emph{direct} method of determining distance to stars is based on
the old concept \emph{parallax} -- the angle between the lines of sight of the
star from two different positions of the observer. The further the observed
object is, the wider the baseline needs to be, to obtain a discernible parallax.
For objects in the solar system, distance points on the Earth can be used. For
stars, the baseline is supplied by the Earth's orbit around the Sun, giving a
distance of $2\unit{au}\approx3\times10^{11}\unit{m}$. The star's position is
determined relative to very distant, fixed stars, at an interval of half a year.
The triangle obtained is very nearly isosceles; the parallax, defined as half
the apex angle, is less than $1''$. The distance to the star is then $d=1/p$, in
astronomical units. In this method, distances of up to 500 light-years can be
measured. The \emph{parsec} is the distance corresponding to a parallax of one
arc second, amounting to about 3 light-years.

\item The surface temperature of a star can be obtained from the shape of its
spectrum -- the \emph{continuum}. The \emph{effective temperature} of a star,
$T_\text{eff}$, is defined as the temperature of a blackbody that would emit the
same radiation flux. This is a good approximation to the temperature of the
star's outermost layer, called the \emph{photosphere}. If the stellar radius is
$R$ then

\begin{equation}\label{eq:LTeff}
L=4\pi{R}^2\sigma{T}_\text{eff}^4.
\end{equation}

The surface temperatures of stars range between a few thousand to a few hundred
thousands degrees Kelvin, the wavelength of maximum radiation shifting from
infrared to X-ray. The effective temperature of the Sun is $5780\unit{K}$.

\item The chemical composition can be inferred from spectral lines, but these
will only reveal elements present in the photosphere, and the deduced
composition is not representative of the bulk. Most of chemical elements were
found present in the spectrum of stars.

\item Under certain conditions, the mass of a star that is a member of a binary
system can be calculated, based on periodic shifts in the spectral lines, that
can be translated into an orbital period. The radius of a star can very rarely
be measured in eclipsing binary systems. More often, the star's radius is
determined from the independently measured luminosity and effective temperature,
using Eq.~\eqref{eq:LTeff}. Stellar masses and radii are measured in units of
\emph{solar mass}, $M_{\Sun}=1.99\times10^{30}\unit{kg}$, and \emph{solar
radius}, $R_{\Sun}=6.96\times10^8\unit{m}$. The mass range is quite narrow --
between $\sim{0.1}$ and a few tens of $M_{\Sun}$. Stellar radii vary typically
between less than $0.01$ to more than $1000 R_{\Sun}$, but more compact stars
exist, with radii of a few tens of kilometers.

\end{itemize}

\subsection{Basic assumptions}
\begin{description}

\item[Isolation] We regard a single star as an isolated body in empty space.
Although a star is always a member of a larger group (a galaxy, a stellar
cluster) this assumption is justified because the typical distance between stars
is on the order of a few light-years. The Sun's closest neighbor (\emph{Proxima
Centauri}) is at a distance of 4.22 light-years. this distance is larger than
the solar diameter by a factor of $\sim{3}\times{10^7}$. The gravitational field
and radiation flux would be diminished by a factor of almost $10^{15}$.

The assumption of isolation means that the course of a star's evolution is
determined only by initial conditions.

\item[Uniform initial composition] A star is born with a given mass and a given
composition. The compositions of stars, as derived from spectral measurements,
are remarkably similar, and are also similar to the composition of the
interstellar medium. As stars are born in the interstellar medium, we may assume
that there is little difference in the \emph{initial} composition of stars.

A newly formed star is composed out of hydrogen (about 70\% by mass), helium
(25--30\%), and traces of other materials, most abundant of which are oxygen,
carbon, and nitrogen. The composition of stellar material is usually described
in \emph{mass fractions} of elements: $X$ denotes the mass fraction of hydrogen,
$Y$ denotes the mass fraction of helium, and $Z$ denotes the mass fraction of
the other elements, so $X+Y+Z=1$.

The assumption of uniform initial composition means that the evolution of a star
is solely dependent on its initial mass -- stars are a one-parameter family.

\item[Spherical symmetry] Departure from spherical symmetry can be caused by
rotation or by the star's own magnetic field. In the majority of cases, the
energy associated with these factors is much smaller than the gravitational
binding energy. The period of the Sun's revolution is about 27 days. This
corresponds to an angular velocity of $\sim{2.7}\times10^{-6}\unit{rad/s}$. The
kinetic energy of rotation relative to the gravitational binding energy is
\begin{equation*}
\frac{M\omega^2R^2}{GM^2/R}=\frac{\omega^2R^3}{GM}\approx2\times10^{-5}.
\end{equation*}
(The spin velocity of other stars can be deduced by the broadening of spectral
lines caused by the Doppler effect.)

The magnetic fields of stars similar to the Sun range from a few thousandths to
a few tenths of a tesla. Magnetic fields can be deduced from splitting of
spectral lines due to the Zeeman effect. The energy density associated with a
magnetic field $B$ is $B^2/2\mu_0$, while the gravitational energy density is
$GM^2/R^2$. For the Sun, taking $B=0.1\unit{T}$ (typical of sunspots) we get
\begin{equation*}
\frac{B^2/\mu_0}{GM^2/R^4}\approx10^{-11}.
\end{equation*}
Compact stars have larger magnetic fields, but their small radii compensate for
this.

Neglecting Magnetic or rotational effects means neglecting also deviations from
spherical symmetry. The physical properties within a star will change as a
function of radial distance only. The spatial variable $r$ can be replaced by
the mass $m$ enclosed within a volume of radius $r$. The transformation is given
in terms of the density $\rho$:
\begin{equation}
dm=\rho\,4\pi{r}^2dr.
\end{equation}
The advantage of using $m$ as the independent variable is that its range is
bounded, ($0\leq{m}\leq{M}$), whereas the radius can change by several orders of
magnitude in the lifetime of a star.

\end{description}

\subsection{The H--R diagram}
The fundamental properties of a star that can be inferred from observation are
its luminosity and effective temperature. Plotting the logarithm of the
luminosity, in solar luminosity units, against the decreasing logarithm of the
effective temperature in degrees Kelvin is known as the H-R diagram (Ejnar
Hertzsprung 1911 and Henry Russell 1913). After populating the H--R diagram with
many observable stars it becomes obvious that stars occupy only certain areas in
the $L$--$T_\text{eff}$ plane. Most stars lie in the diagonal strip known as the
\emph{main sequence}. Another populated area is to the right and above of the
main sequence. Stars in this region are brighter than main sequence stars of the
same temperature, or put another way, they are cooler than main sequence stars
of the same brightness. For this reason they are called \emph{red giants}. A
third populated region lies to the left and below the main sequence. Stars in
this region are then hotter than main sequence stars of the same brightness, and
are therefore called \emph{white dwarfs} (not dwarves).

Looking at H--R diagrams of different star clusters it is possible to learn
something about the evolution of stars. Stars within a cluster are assumed to
have formed at roughly the same time, by fragmentation of a gas cloud, and their
diagram is therefore a kind of snapshot picture of a moment in the life of the
group. From several of these snapshots we can deduce that being on or outside
of the main sequence is a determined by age, whereas the location along the main
sequence is determined `at birth', i.e., determined by initial mass only.

Choosing main sequence stars with known masses and looking for a correlation
between their masses and luminosities, an empirical relation,
\begin{equation}
\frac{L}{L_{\Sun}}=\left(\frac{M}{M_{\Sun}}\right)^\nu,
\end{equation}
is found, with $\nu$ between 3 and 5. (The Sun is a main-sequence star.)

\section{The equations of stellar evolution}

\subsection{Local thermodynamic equilibrium}
A star is a gaseous sphere, made of three kinds of particles -- electrons,
protons and heavier nuclei, and the `particles' of electromagnetic radiation,
photons. (Atoms and molecules cannot survive long in the energetic environment.)
Frequent collision between all kinds of particles result in a state of
thermodynamic equilibrium characterized by a single temperature. The kinetic
energies of the electrons and protons are distributed according to Maxwell's
formula, and the energies of the photons are distributed according to Planck's
function, with the same temperature.

The mean free path of all particles is many orders of magnitude smaller than the
geometric dimensions of a star, and the mean free time is likewise smaller than
the timescale for macroscopic changes in the star. This means that the star is
also in \emph{local thermodynamic equilibrium} (LTE), i.e., any part of the star
is characterized by a single temperature, although the temperature (and thus
other properties) is not uniform in space nor constant in time.

The \emph{structure} of a star of mass $M$ is uniquely determined if the
temperature $T$, the density $\rho$, and the composition (mass fractions), are
known at every point in the star. (A ``point'' is a value of $r$ or $m$.) The
\emph{evolution} of a star composed of $n$ different elements is described by
$n+2$ functions: $\rho(m,t)$, $T(m,t)$, and $X_i(m,t)$. We will now derive a
set of simultaneous equations that connect these functions.

\subsection{The energy equation}
Consider a small element of mass $dm$ within a star, a thin spherical shell
between radii $r$ and $r+dr$. Let $u$ be the internal energy per unit mass and
$P$ the pressure. Then, according to the first law of thermodynamics
\begin{equation}\label{eq:energy}
\delta(udm) = dm\delta{u} = \delta{Q} + \delta{W}.
\end{equation}
The work $\delta{W}$ can be expressed as
\begin{equation}\label{eq:W}
\delta{W} = -P\delta{d}V = -P\delta\left(\frac{dV}{dm}\,dm\right) =
-P\delta\Bigl(\frac{1}{\rho}\Bigr)dm.
\end{equation}
The sources of heat of the mass element are the release of nuclear energy, and
heat flow in and out of the element. The rate of energy release through nuclear
reactions, per unit mass, is denoted by $q$, and the total heat flow per unit
time is denoted by $F$. Then
\begin{equation}\label{eq:Q}
\delta{Q} = qdm\delta{t} + F(m)\delta{t} - F(m+dm)\delta{t} =
\left(q - \frac{\di{F}}{\di{m}}\right)dm\delta{t}.
\end{equation}
Substituting~\eqref{eq:W} and~\eqref{eq:Q} into~\eqref{eq:energy} we get
\begin{equation}
dm\delta{u} + P\delta\left(\frac{1}{\rho}\right)dm =
\left(q - \frac{\di{F}}{\di{m}}\right)dm\delta{t},
\end{equation}
and in the limit $\delta{t}\to\infty$,
\begin{equation}\label{eq:energy2}
\dot{u} + P\dot{\left(\frac{1}{\rho}\right)} = q - \frac{\di{F}}{\di{m}}.
\end{equation}

If temporal derivatives vanish then
\begin{equation}
q = \frac{dF}{dm},
\end{equation}
which is equivalent to
\begin{equation}
L_\text{nuc} = \int_0^Mq\,dm,
\end{equation}
i.e., the energy is radiated away from the star at the same rate is is produced
in the interior.

\subsection{The equation of motion}
Consider a small cylindrical volume element within a star, with an axis of
length $dr$ in the radial direction, between $r$ and $r+dr$, and a base $dS$.
The mass is given by
\begin{equation}\label{eq:delem}
\Delta{m} = \rho{d}rdS.
\end{equation}

The forces acting on this element are the force of gravity and the pressure
gradient. The equation of motion (Newton's second law) is expressed by
\begin{equation}
\ddot{r}\Delta{m} = -\frac{Gm\Delta{m}}{r^2} + P(r)dS - P(r+dr)dS.
\end{equation}
Substituting $\Delta{m}$ from~\eqref{eq:delem} we get
\begin{equation}\label{eq:eom1}
\ddot{r} = -\frac{Gm}{r^2} - \frac{1}{\rho}\frac{\di{P}}{\di{r}}.
\end{equation}
Making the transformation to $m$ as the independent variable,
Eq.~\eqref{eq:eom1} becomes
\begin{equation}\label{eq:eom2}
\ddot{r} = -\frac{Gm}{r^2} - 4\pi{r}^2\frac{\di{P}}{\di{m}}.
\end{equation}

In a state of \emph{hydrostatic equilibrium}, with gravitational and pressure
forces exactly in balance,
\begin{equation}
\frac{\di{P}}{\di{r}} = -\rho\frac{Gm}{r^2},
\end{equation}
or,
\begin{equation}\label{eq:hydro2}
\frac{\di{P}}{\di{m}} = -\frac{Gm}{4\pi{r}^4}.
\end{equation}
Note that hydrostatic equilibrium implies that pressure drops outward.

We can estimate the pressure at the center of a star by integrating
Eq.~\eqref{eq:hydro2} from the center to the surface,
\begin{equation}
P(M) - P(0) = -\int_0^M\frac{Gm}{4\pi{r}^4}\,dm.
\end{equation}
We can thus obtain a lower limit on the central pressure $P_c$:
\begin{equation}
P_c = \int_0^M\frac{Gm}{4\pi{r}^4}\,dm>\int_0^M\frac{Gm}{4\pi{R}^4}dm =
\frac{GM^2}{8\pi{R}^4}.
\end{equation}
In solar units,
\begin{equation}
P_c > 4.48\times10^{13}\left(\frac{M}{M_{\Sun}}\right)^2\left(\frac{R_{\Sun}}
{R}\right)^4\unit{N\,m^{-2}}.
\end{equation}
The pressure at the center of the Sun exceeds $4\times10^8$ atmospheres, or 400
mega bars.

\subsection{The virial theorem}
An important consequence of hydrostatic equilibrium is a link between
gravitational potential energy and internal energy (or kinetic energy in
mechanics). Multiplying the equation of hydrostatic equilibrium,
Eq.~\eqref{eq:hydro2}, by $\frac{4}{3}\pi{r}^3$ and integrating over the whole
star, we get
\begin{equation}\label{eq:virial}
-3\int_0^M\frac{P}{\rho}\,dm = -\int_0^M\frac{Gm}{r}\,dm.
\end{equation}
The integral on the right hand side is just the gravitational potential energy,
$\Omega$. This is the global form of the virial theorem.

Consider now the case of an ideal gas. The gas pressure is given by
$P = (\rho/m_g)kT$, where $m_g$ is the mass of a gas particle and $k$ is the
Boltzmann constant. The internal energy per unit mass is given by
\begin{equation}\label{eq:yukatan}
u = \frac{3}{2}\frac{kT}{m_g} = \frac{3}{2}\frac{P}{\rho}.
\end{equation}
Combining \eqref{eq:yukatan} with~\eqref{eq:virial} we have
\begin{equation}
\int_0^Mu\,dm = -\frac{1}{2}\Omega.
\end{equation}
The integral on the left hand side is just the total internal energy $U$, and
thus
\begin{equation}
U = -\frac{1}{2}\Omega.
\end{equation}

We can use the last result to estimate the average internal temperature of a
star, with the (justifiable) assumption that stellar material behaves as an
ideal gas. The gravitational potential energy of a star of mass $M$ and radius
$R$ is $\Omega = -\alpha{G}M^2/R$, with $\alpha$ a constant on the order of
unity determined by the density profile. By the virial theorem we have
$U = \frac{1}{2}\alpha{G}M^2/R$. On the other hand, from Eq.~\eqref{eq:yukatan},
we have
\begin{equation}
U = \int_0^M\frac{3}{2}\frac{kT}{m_g}\,dm=\frac{3}{2}\frac{k}{m_g}\bar{T}M.
\end{equation}
We get, for the average temperature $\bar{T}$,
\begin{equation}\label{eq:tibar}
\bar{T} = \frac{\alpha{m}_gG}{3k}\frac{M}{R},
\end{equation}
or
\begin{equation}
\bar{T}\propto{M}^{2/3}\bar{\rho}^{1/3}.
\end{equation}
So between two stars of the same mass, the denser one is also the hotter. To get
an order of magnitude on the average temperature we'll take $\alpha=0.5$ and
assume a gas of atomic hydrogen, to come up with
\begin{equation}
\bar{T} \approx 4\times10^6\left(\frac{M}{M_{\Sun}}\right)\left(\frac{R_{\Sun}}
{R}\right)\unit{K}.
\end{equation}
Note that the average temperature is much higher than the observed temperature
of the photosphere, implying that the internal temperatures must reach higher
values still. At millions of degrees Kelvin, hydrogen, helium, and most all
elements are ionized completely. Stellar material must therefore be
\emph{plasma}.

\subsection{The total energy of a star}
\begin{itemize}
\item Using the energy equation~\eqref{eq:energy2} and the equation of
motion~\eqref{eq:hydro2}, and integrating over the whole star, we can show that
\begin{equation}\label{eq:totalenergy}
\dot{U} + \dot{\Omega} + \dot{{K}} = L_\text{nuc} - L,
\end{equation}
where $K$ is the kinetic energy of expansion,
$K = \int_0^M\frac{1}{2}\dot{r}^2\,dm$.

\item If the star is in thermal equilibrium ($\dot{E}=0$) \emph{and} in
hydrostatic equilibrium ($K=\dot{K}=0$), then the sum of internal and
gravitational energies must be conserved. But since in hydrostatic equilibrium
$U$ and $\Omega$ are related by the virial theorem, it follows that each must be
conserved independently. For example, although it is possible for a star to
expand or contract without violating hydrostatic equilibrium, a star in thermal
and hydrostatic equilibrium cannot cool and expand and the same time,
maintaining the total energy.

\item Another conclusion that can be made is that a star in hydrostatic
equilibrium has a negative `heat capacity' -- it becomes hotter when losing
energy. Recall that
\begin{equation}
E = U + \Omega = \frac{1}{2}\Omega = -U.
\end{equation}
Thus if $\dot{E}<0$ then $U$, and consequently $\bar{T}$ must increase. This is
accomplished by the star contracting (without violating hydrostatic
equilibrium), supplying enough energy to raise the mean temperature.

\item Ideal gas was assumed throughout.
\end{itemize}

\subsection{The equations governing composition changes}
\begin{itemize}
\item Stellar material is composed of free electrons and bare, chemically
unbound nuclei, so composition changes cannot occur by chemical reactions. The
possible changes in abundances of the constituents are by transformation from
one element into another -- by nuclear reactions.

\item Denote by $\rho_i$ the partial density of the $i$th atomic species, so
that $X_i=\rho_i/\rho$. To a good approximation the number density of the
species would be
\begin{equation}
n_i = \frac{\rho_i}{A_im_H} = \frac{\rho}{m_H}\frac{X_i}{A_i},
\end{equation}
where $A_i$ is the \emph{baryon number} and $m_H$ the \emph{atomic mass unit}.

\item Nuclear reactions are subject to conservation laws. The number of baryons
must be conserved, the number of leptons must be conserved, and of course, the
charge must be conserved.

\item The rate of reactions depends on temperature and density, and on the
particle's interaction cross-section. Introducing a composition `vector', we can
write the composition change equations as
\begin{equation}
\dot{\V{X}} = \V{f}(\rho,T,\V{X}).
\end{equation}
\end{itemize}

\subsection{The set of evolution equations}
\begin{itemize}
\item The set of nonlinear partial differential equations describing the
evolutionary course of a star is
\begin{subequations}
\begin{align}\label{eq:set}
\ddot{r} &= -\frac{Gm}{r^2}-4\pi{r}^2\frac{\di{P}}{\di{m}},\\
\dot{u} &+ P\dot{\biggl(\frac{1}{\rho}\biggr)} = q - \frac{\di{F}}{\di{m}},\\
\dot{\V{X}} &= \V{f}(\rho,T,\V{X}).
\end{align}
\end{subequations}

\item The unknowns are the structure functions -- $\rho(m,t)$, $T(m,t)$, and
$\V{X}(m,t)$. The independent variable is $m$, and we have the static relations
$m(r)=\int_0^r4\pi{}r^2\rho(r)\,dr$ and $dm=4\pi\rho{}r^2dr$ hold at any time.

\item The equations contain other functions, that need to be supplied in terms
of the unknowns. Thermodynamics or statistical mechanics can supply $P$ and $u$;
atomic physics and radiation transfer will be invoked for $F$; and nuclear and
particle physics for $q$ and $\V{f}$.

\item Boundary and initial conditions have to be supplied. Boundary conditions
are straightforward: $P(M,t)=0$ and $F(0,t)=0$. The initial state of a star is a
rather difficult question that will need to be circumvented.
\end{itemize}

\subsection{The characteristic timescales of stellar evolution}
The evolution of a star is described by three time-dependent equations, dealing
with three different types of physical changes: dynamical, or structural change
in the first equation, thermal changes in the second, and changes in
composition, resulting from nuclear reactions, in the third. Each of these
processes has a different characteristic time scale.

\subsubsection*{The dynamical timescale}
\begin{itemize}
\item The characteristic dimension of the star is the radius $R$. The
characteristic velocity in a gravitational field is the escape velocity
$v\sub{esc}=\sqrt{2GM/R}$. The characteristic timescale of dynamical processes
is
\begin{equation}
\tau\sub{dyn} = \frac{R}{v\sub{esc}} =
\sqrt{\frac{R^3}{2GM}} \approx \sqrt{\frac{1}{G\bar{\rho}}}.
\end{equation}

\item The dynamical timescale of the Sun is roughly $1000\unit{s}$, and
generally
\begin{equation}
\tau\sub{dyn} \approx
1000\sqrt{\left(\frac{R}{\rsun}\right)^3\left(\frac{\msun}{M}\right)}
\unit{s}.
\end{equation}

\item The dynamical timescale is very short, many orders of magnitudes shorter
than the typical stellar lifetime.

\item A dynamical process occurs when the gravitational force is not balanced
by the pressure. This can result in a catastrophic event -- collapse or
explosion, or in return to hydrostatic equilibrium. Either way, the end state
will be achieved in a time comparable to the dynamical time scale.

\item If a star cannot recover from a dynamical process by restoring hydrostatic
equilibrium, the resulting explosion should be observable in its entirety --
this is a supernova.

\item A star that is not currently exploding or collapsing may be assumed to be
in a state of quasi-hydrostatic equilibrium. Any perturbation is quickly
quenched. Consequently, the virial theorem may be assumed to hold at all times.
\end{itemize}

\subsubsection*{The thermal timescale}
\begin{itemize}
\item The internal energy of the star, $U$, is the characteristic thermal
property. By the virial theorem $U\approx{G}M^2/R$. The characteristic rate of
change of $U$ is the rate at which the star radiates energy, $L$. The thermal
timescale is then
\begin{equation}
\tau\sub{th} = \frac{U}{L} \approx \frac{GM^2}{RL}.
\end{equation}

\item The thermal timescale for the Sun is $10^{15}\unit{s}$, and generally
\begin{equation}
\tau\sub{th} =
10^{15}\left(\frac{M}{\msun}\right)^2\left(\frac{\rsun}{R}\right)
\left(\frac{\lsun}{L}\right)\unit{s}.
\end{equation}

\item The thermal timescale is many orders of magnitude longer than the
dynamical timescale, but is still only a fraction of the lifespan of the star.
We may assume that throughout most of its life a star maintains thermal
equilibrium.

\item If a star maintains thermal and hydrostatic equilibrium its energy is
conserved, and by the virial theorem its internal energy and gravitational
energy are both independently conserved. This means that under these
circumstances if a part of the star contracts another part must expand; if the
temperature rises in some place it must decrease in another.

\item That the age of the Earth is evidently, from biology and geology, much
longer that the Sun's thermal timescale was historically a decisive clue that
reservoirs or energy exists in subatomic process maintaining the Sun's
luminosity much longer than would be possible through gravitational contraction
alone.
\end{itemize}

\subsubsection*{The nuclear timescale}
\begin{itemize}
\item The quantity that is changed by nuclear processes, besides abundances, is
a small fraction of the rest-mass energy given by $E=mc^2$. Call this the
nuclear potential energy. The rate of change of nuclear potential energy is the
nuclear luminosity $L\sub{nuc}$. Since we assume thermal equilibrium
$L\sub{nuc}=L$ and
\begin{equation}
\tau\sub{nuc} = \frac{\eps{M}c^2}{L},
\end{equation}
where $\eps$ is a small fraction estimated by the typical binding energy of a
nucleon divided by the nucleon's rest-mass energy, which amounts to a few
$10^{-3}$.

\item Substituting solar values,
$\tau\sub{nuc}\approx{5}\times{10^{17}}\unit{s}$. This is larger than the Sun's
age (or even the age of the universe).

\item Throughout its life a star consumes only a fraction of the available
nuclear energy, and only a fraction of the star has changed its original
composition.

\item A star may be assumed to be in thermal and hydrostatic equilibrium in each
\emph{stage} of its evolution, the stages being determined by the nuclear
processes. At each stage, the set of evolution equations reduce to
\begin{subequations}
\begin{align}\label{eq:set2}
\frac{dP}{dm} &= -\frac{Gm}{4\pi{r}^4},\\
\frac{dF}{dm} &= q,\\
\dot{\V{X}} &= \V{f}(\rho,T,\V{X}).
\end{align}
\end{subequations}

\item With the reduced set we need not know the initial structure of the star,
only the initial composition, which we assume to be similar to the interstellar
medium. To trace the evolution of a star we need to know (a) what is the
sequence of nuclear processes, and (b) given a fixed composition, what is the
structure of a star in thermal and hydrostatic equilibrium.
\end{itemize}

\section{Elementary physics of gas and radiation in stellar interiors}

\subsection{The equation of state}
\begin{itemize}
\item At the temperatures prevailing in stars, gases are ionized and Coulomb
interactions are expected. Nevertheless, we treat the mixture of free electrons
and ions as a non-interacting gas. This is justified because the energy
associated with the Coulomb interactions is much smaller than the typical
kinetic energy of the particles. The typical inter-particle distance is
\begin{equation*}
d \approx \left(\frac{Am_H}{\bar{\rho}}\right)^{1/3} =
\left(\frac{4\pi{A}m_H}{3M}\right)^{1/3}R.
\end{equation*}
The energy per particle of Coulomb interactions is therefore
\begin{equation*}
\eps_C \approx \frac{1}{4\pi\eps_0}\frac{Z^2e^2}{d}
\end{equation*}
and the ratio
\begin{equation*}
\frac{\eps_C}{k\bar{T}} \approx
\frac{1}{4\pi\eps_0}\frac{Z^2e^2}{A^{4/3}m^{4/3}GM^{2/3}}
\end{equation*}
(using eq.~\ref{eq:tibar} and with $Z=1$, $A=1$) is less than 0.01 for
$M=M_{\Sun}$, remaining small, even for heavier materials (because $A\approx2Z$)
until $M\lesssim{}10^{-3}\msun{}$. Note that Jupiter-like planets are in this
mass range and therefore require more careful consideration of the equation of
state.

\item To calculate the pressure of a free particle system we use the
\href{run:./pressure_integral.pdf}{pressure integral}:
\begin{equation}
P=\frac{1}{3}\int_0^\infty{v}p\,n(p)\,dp.
\end{equation}

\item The pressure of a mixture of particles of different species is given by
the sum of the partial pressures. In stars $P$ will be the sum of three terms:
$P_I$ for the ions, $P_e$ for the electrons, and $P_\text{rad}$ for the photons.

\item We denote by $P_\text{gas}$ the contribution of both ions and electrons,
and $\beta$ is the fractional contribution from $P\sub{gas}$.
\end{itemize}

\end{document}
