\documentclass[10pt,amsmath,amssymb,aps,pra]{revtex4-2}

\newcommand{\V}[1]{\mathbf{#1}} % bold vector or matrix
\newcommand{\sub}[1]{_{\text{#1}}} % text subscript
\newcommand{\di}{\partial} % partial derivative
\newcommand{\unit}[1]{\ \mathrm{#1}} % attach units
\newcommand{\kros}{\times} % vector product
\newcommand{\grad}{\boldsymbol{\nabla}} % gradient operator
\newcommand{\divergenz}{\boldsymbol{\nabla}\cdot} % divergence operator
\newcommand{\curl}{\boldsymbol{\nabla}\kros} % curl (rot) operator
\newcommand{\rom}[1]{\mathrm{#1}} % upright function name
\newcommand{\abs}[1]{\left\vert#1\right\vert} % generic absolute value
\newcommand{\set}[1]{\left\{#1\right\}} % put elements between { }
\newcommand{\Real}{\mathbb R} % Real numbers field
\newcommand{\Cmplx}{\mathbb C} % Complex numbers field
\newcommand{\eps}{\varepsilon} % variirtes epsilon
\newcommand{\mean}[1]{\langle #1 \rangle} % physics mean
\newcommand{\about}{\sim\!}

\begin{document}

\title{The pressure integral}
\author{Naor Movshovitz}
\affiliation{UC Santa Cruz}

\begin{abstract}
Derivation of the pressure integral. From Clayton.
\end{abstract}

\maketitle

The microscopic source of pressure in a \emph{perfect gas} is particle
bombardment. The reflection of these particles from a real or imagined surface
inside the gas results in transfer of momentum to that surface. By Newton's
second law, that momentum transfer exerts a force on the surface. The average
force per unit area is called the pressure.

In \emph{thermal equilibrium} the angular distribution of particle momenta
should be isotropic. Imagine a surface inside the gas with a normal
$\V{\hat{n}}$. If a particle of momentum $\V{p}$, inclined at an angle $\theta$
to $\V{\hat{n}}$, is \emph{specularly reflected}  from the surface, the momentum
transferred is normal to the surface and equals
$\Delta{p\sub{n}}=2p\cos{\theta}$. Let $F(\theta,p)\,d\theta\,d{p}$ be the
number of particles with momentum magnitude $p$ in the range $dp$ striking the
surface per unit area per unit time from all directions inclined at angle
$\theta$ in the range $d\theta$ to the normal. The contribution to the pressure
from these particles is
\begin{equation}
dP = 2p\cos{\theta}\,F(\theta,p)\,d\theta\,dp.
\end{equation}
The total pressure then is
\begin{equation}
P = \int_{\theta=0}^{\pi/2}\int_{p=0}^{\infty}
2p\cos{\theta}\,F(\theta,p)\,d\theta\,dp.
\end{equation}

In thermodynamic equilibrium, the angular distribution of momenta is isotropic.
The distribution of momenta magnitudes is given by statistical mechanics. The
flux $F(\theta,p)\,d\theta\,dp$ may be written as the product of the number
density of particles with momentum in the given range times the volume of such
particles capable of striking the unit surface in unit time. This volume is
equal to $\cos{\theta}$ times the velocity associated with momentum of magnitude
$p$, denoted $v_p$. That is,
\begin{equation}
F(\theta,p)\,d\theta\,dp = v_p\cos{\theta}\,n(\theta,p)\,d\theta\,dp,
\end{equation}
where $n(\theta,p)\,d\theta\,dp$ is the number density of particles with these
momenta. Because of isotropy the fraction of particles moving in the cone of
directions is the fraction of solid angle subtended by this cone. So if
$n(p)\,dp$ is the total number density of particles with momentum magnitude $p$
in $dp$ then
\begin{equation}
\frac{n(\theta,p)\,d\theta\,dp}{n(p)\,dp} =
\frac{2\pi\sin{\theta}\,d\theta}{4\pi}.
\end{equation}
The gas pressure is therefore
\begin{equation}
P = \int_{0}^{\pi/2}\int_{0}^{\infty}
2p\cos{\theta}\,v_p\cos{\theta}\,n(p)\,dp\,\frac{1}{2}\sin{\theta}\,d\theta.
\end{equation}
The integration over angles yields
\begin{equation}\label{eq:presint}
P = \frac{1}{3}\int_{0}^{\infty}pv_pn(p)\,dp.
\end{equation}

Equation~(\ref{eq:presint}) is known as the \emph{pressure integral}. It is
valid for a perfect isotropic ``gas''. The relationship of $v_p$ to $p$ depends
on relativistic considerations. The distribution $n(p)$ depends on the type of
particles and can require quantum statistical mechanics.

\end{document}
